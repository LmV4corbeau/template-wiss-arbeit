\usepackage[utf8]{inputenc}
\usepackage{graphicx}
\usepackage[parfill]{parskip}
\usepackage[ngerman]{babel}
\usepackage[T1]{fontenc}
\usepackage[onehalfspacing]{setspace}
\usepackage{tabularx}

\usepackage{lipsum}%Zum Einfügen eines Lorem Ipsum Textes

\usepackage{enumitem} %Um Nummerieungen durch Text unterbrechen zu können
\usepackage{pdfpages}% Zum Einfügen mehrseitiger PDF-Dokumente

\usepackage[left=3cm,right=2cm,top=2cm,bottom=2cm,includehead]{geometry}%Maße für die wissenschaftliche Arbeit

% Ab hier werden die Punkte für Chapter hinzugefügt
\usepackage{titletoc}

\titlecontents{chapter}[1.5em]{\addvspace{1pc}\normalfont\sffamily\bfseries}{\contentslabel{1.5em}}
{\hspace*{-1.5em}}{\hspace*{0.2em} \titlerule*[0.8pc]{.}\contentspage}

\usepackage{glossaries}%Zum Erzeugen des Glossars 

% enthält die konfiguration für das listings package
\usepackage{listings}
\usepackage{xcolor}

\lstset{language=Java}

\definecolor{lst_light_grey}
{rgb}{0.95,0.95,0.95}

\definecolor{lst_dark_grey}
{rgb}{0.8,0.8,0.8}

\definecolor{lst_highlight}
{rgb}{0,0,0.6}

\definecolor{lstgreen}
{rgb}{0,0.6,0}

\definecolor{lstmauve}
{rgb}{0.58,0,0.82}

\lstset{ %
	basicstyle=\small\ttfamily, %
	backgroundcolor=\color{lst_light_grey}, %
	captionpos=b, %
 	commentstyle=\color{lstgreen}, %
 	frame=single, %
	tabsize=2, %
%
	keywordstyle=\color{lst_highlight}, %
%
 	numbers=left, %
%
 	numberstyle=\scriptsize \color{lst_dark_grey}, %
%
 	rulecolor=\color{lst_dark_grey}, %
}

% additional highlighted keywords
\lstset { emph= {%
	var, function %
	}, emphstyle={\color{lst_highlight}}%
}






